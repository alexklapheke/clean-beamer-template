% LaTeX Beamer template, made with linguistics in mind
% 
% * Intended to be compiled with XeLaTeX: <http://www.xelatex.org/>
% * Uses the free Libertine font: <http://www.linuxlibertine.org/>
%   (defines character U+203A in top bar)
% * This file calls packages: booktabs, fontspec, gb4e, natbib, newtxmath,
%                             tabularx, tikz, tikz-qtree
% 
% Feel free to use or modify this for any purpose, but please keep the following
% attribution and license in the source code. Attribution is not required in the
% compiled PDF.


% The MIT License (MIT)
% 
% Copyright (c) 2014 Alexander Klapheke
% 
% Permission is hereby granted, free of charge, to any person obtaining a copy
% of this software and associated documentation files (the "Software"), to deal
% in the Software without restriction, including without limitation the rights
% to use, copy, modify, merge, publish, distribute, sublicense, and/or sell
% copies of the Software, and to permit persons to whom the Software is
% furnished to do so, subject to the following conditions:
% 
% The above copyright notice and this permission notice shall be included in all
% copies or substantial portions of the Software.
% 
% THE SOFTWARE IS PROVIDED "AS IS", WITHOUT WARRANTY OF ANY KIND, EXPRESS OR
% IMPLIED, INCLUDING BUT NOT LIMITED TO THE WARRANTIES OF MERCHANTABILITY,
% FITNESS FOR A PARTICULAR PURPOSE AND NONINFRINGEMENT. IN NO EVENT SHALL THE
% AUTHORS OR COPYRIGHT HOLDERS BE LIABLE FOR ANY CLAIM, DAMAGES OR OTHER
% LIABILITY, WHETHER IN AN ACTION OF CONTRACT, TORT OR OTHERWISE, ARISING FROM,
% OUT OF OR IN CONNECTION WITH THE SOFTWARE OR THE USE OR OTHER DEALINGS IN THE
% SOFTWARE.


\documentclass[xetex,serif,xcolor=x11names,compress]{beamer}
\usetheme{clean}

% Sets title, author, etc.
\title[Title]{Title of My Presentation}
\subtitle{Subtitle}
\author[Klapheke]{Alexander~Klapheke}
\institute[Harvard]{Harvard~University}
\date{November 14, 2012}

% font
\usepackage{fontspec}
\usepackage[libertine]{newtxmath}
\setromanfont[Mapping=tex-text,Numbers=Lining]{Linux Libertine O}
\setmonofont[Scale=MatchLowercase]{Inconsolata Plus 0}

% Strongly discourage hyphenation
\hyphenpenalty=5000
\tolerance=1000

% citations
\usepackage[round,sort&compress]{natbib}

% nice-looking tables
\RequirePackage{booktabs}
% tables with relative fixed-width. use code below to allow wrapping
\RequirePackage{tabularx}
\newcolumntype{C}{>{\centering\arraybackslash}X}
\newcolumntype{L}{>{\raggedright\arraybackslash}X}

% highlighting
\newcommand\hla[1]{\textcolor{CustomRed}{#1}}  % primary color
\newcommand\hlb[1]{\textcolor{CustomBlue}{#1}} % secondary color

%%%%%%%%%%%%%%%%%%%%%%%%%%%%%%%%%%%%%%%
% Linguistics-specific things follow: %
%%%%%%%%%%%%%%%%%%%%%%%%%%%%%%%%%%%%%%%

% glosses
\usepackage{gb4e}
\primebars % use primes instead of 
\let\eachwordone=\it % italicize original in glosses
\resetcounteronoverlays{exx} % don't increment example numbers on \pause

% for arrows in glosses
\newcommand{\tikzmark}[1]{\tikz[overlay,remember picture] \node (#1) {};}

% for trees and graphs
\usepackage{tikz,tikz-qtree}

% Uncomment to start in presentation mode automatically
\hypersetup{%
	colorlinks=false,
	% pdfpagemode=FullScreen,
}

\begin{document}

\section{}
\begin{frame}
	\titlepage
\end{frame}

\section{Section}
\subsection{Formatting}
\begin{frame}{Formatting}
	\vfill

	Lorem ipsum dolor sit amet, \alert{consectetur adipisicing elit}, sed do eiusmod tempor incididunt ut labore et dolore magna aliqua. \citep{aspects}

	\vfill

	\begin{block}{This is a Block}
		This is important information
	\end{block}

	\vfill
	\begin{itemize}
		\item $\left\llbracket\textsc{Imperf}\right\rrbracket=\mathrm{λ}P_{\langle l,st\rangle}\,.\,\mathrm{λ}t\,.\,\mathrm{λ}w\,.\,\exists e\left[t\subseteq \mathrm{τ}(e)\land P(e,w)=1\right]$

		\item
			John \tikz[baseline=(a.base),remember picture]\node(a){\hla{can play}}; the guitar, and
			Mary \tikz[baseline=(b.base),remember picture]\node(b){\hla{Δ}}; the violin.
			\begin{tikzpicture}[overlay,remember picture,>=latex]
				\draw[semithick,->,out=315,in=225,looseness=0.5,color=CustomRed] (a.south) to (b.south);
			\end{tikzpicture}

	\end{itemize}

	\vfill

\end{frame}

\subsection{Gloss with IPA}
\begin{frame}{Gloss with IPA}
	\begin{exe}
		\ex{\label{ex:french} {\glll Sont des mots qui vont très bien ensemble. \\
				sɔ̃ de mo ki vɔ̃ tʁɛ bjɛ̃ ɑ̃sɑ̃bl \\
				be.\textsc{3p} \textsc{part} word-\textsc{pl} \textsc{rel.nom} go.\textsc{3pl} very well together \\
			\trans `These are words that go together well.'}
		}
	\end{exe}
\end{frame}

\subsection{Table}
\begin{frame}{Table}
	\newcommand\tablep{\hlb{\Large\textbf{+}}}
	\newcommand\tablem{\hla{\Large\textbf{−}}}
	% \newcommand\tablep{\hlb{\large ✔}}
	% \newcommand\tablem{\hla{\large ✘}}
	\begin{tabularx}{\linewidth}{lCCCC} \toprule
		Verb type      & Stative & Durative & Telic   & Subinterval \\ \midrule
		State          & \tablep & \tablep  & \tablem & \tablep     \\
		Activity       & \tablem & \tablep  & \tablem & \tablep     \\
		Accomplishment & \tablem & \tablep  & \tablep & \tablem     \\
		Achievement    & \tablem & \tablem  & \tablep & \tablem     \\
		Semelfactive   & \tablem & \tablem  & \tablem & \tablem     \\ \bottomrule
	\end{tabularx}
\end{frame}

\subsection{Tree}
\begin{frame}{Tree}
	\begin{center}
		\begin{tikzpicture}
			\tikzset{every tree node/.style={align=center,anchor=north}}
			\Tree [.TP
				[.DP \edge[roof]; \node(s){\hlb{Socrates}}; ]
				[.T$'$
					[.T \textit{pres} ]
					[.VP
						[.\node(t){\hla{$t$}}; ]
						[.V$'$
							[.V \hlb{is} ]
							[.AdjP \edge[roof]; \hlb{mortal} ]
						]
					]
				]
			]
			\hla{\draw[semithick,->,out=225,in=270,looseness=1.25,shorten >=1pt,shorten <=1pt] (t.south west) to (s.south);}
		\end{tikzpicture}
	\end{center}
\end{frame}

\subsection{Graph}
\begin{frame}{Graph}
	\begin{center}
		\begin{tikzpicture}[thick,color=CustomPurple]
			% Axes
			\draw[->] (0,0) -- coordinate (x axis mid)(9.5,0);
			\draw[->] (0,0) -- coordinate (y axis mid)(0,5.5);
			
			% Tick marks
			\foreach \x in {0,...,9}
				\draw (\x cm,1pt) -- (\x cm,-3pt) node[anchor=north] {\x};
			\foreach \y in {0,...,5}
				\draw (1pt,\y cm) -- (-3pt,\y cm) node[anchor=east] {\y};

			% Labels
			\node[below=0.5cm] at (x axis mid) {Time};
			\node[rotate=90,above=0.5cm] at (y axis mid) {Data};

			% Data
			\draw[fill=CustomLightBlue] plot[ybar,bar width=16pt]
				coordinates{(1,0.25) (2,1) (3,2) (4,2.25) (5,2.75) (6,3.25) (7,3.25) (8,3.75) (9,4.5)};
			\draw[color=CustomRed,ultra thick] plot[smooth]
			coordinates{(0,0) (0.5,0.125) (1.5,0.625) (2.5,1.5) (3.5,2.125) (4.5,2.5) (5.5,3) (6.5,3.25) (7.5,3.5) (8.5,4.125) (9.5,4.75)};
		\end{tikzpicture}
	\end{center}
\end{frame}

\section*{References}
\begin{frame}{References}
	% Can replace following with: \bibliography{presentation}
	\begin{thebibliography}{1}
		\bibitem[{Chomsky(1965)}]{aspects}
			Chomsky, N. 1965.
			\newblock \textit{Aspects of the Theory of Syntax}.
			\newblock Cambridge, MA: MIT Press
	\end{thebibliography}
\end{frame}

\section{}
\begin{frame}
	\begin{center}
		{\Huge Thank you!}
	\end{center}
\end{frame}

\end{document}
