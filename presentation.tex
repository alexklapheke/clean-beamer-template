% LaTeX Beamer template, made with linguistics in mind
% 
% * Intended to be compiled with XeLaTeX: <http://www.xelatex.org/>
% * Uses the free Libertine font: <http://www.linuxlibertine.org/>
%   (defines character U+203A in top bar)
% * This file calls packages: booktabs, fontspec, gb4e, natbib, newtxmath,
%                             tabularx, tikz, tikz-qtree
% 
% Feel free to use or modify this for any purpose, but please keep the following
% attribution and license in the source code. Attribution is not required in the
% compiled PDF.


% The MIT License (MIT)
% 
% Copyright (c) 2014 Alexander Klapheke
% 
% Permission is hereby granted, free of charge, to any person obtaining a copy
% of this software and associated documentation files (the "Software"), to deal
% in the Software without restriction, including without limitation the rights
% to use, copy, modify, merge, publish, distribute, sublicense, and/or sell
% copies of the Software, and to permit persons to whom the Software is
% furnished to do so, subject to the following conditions:
% 
% The above copyright notice and this permission notice shall be included in all
% copies or substantial portions of the Software.
% 
% THE SOFTWARE IS PROVIDED "AS IS", WITHOUT WARRANTY OF ANY KIND, EXPRESS OR
% IMPLIED, INCLUDING BUT NOT LIMITED TO THE WARRANTIES OF MERCHANTABILITY,
% FITNESS FOR A PARTICULAR PURPOSE AND NONINFRINGEMENT. IN NO EVENT SHALL THE
% AUTHORS OR COPYRIGHT HOLDERS BE LIABLE FOR ANY CLAIM, DAMAGES OR OTHER
% LIABILITY, WHETHER IN AN ACTION OF CONTRACT, TORT OR OTHERWISE, ARISING FROM,
% OUT OF OR IN CONNECTION WITH THE SOFTWARE OR THE USE OR OTHER DEALINGS IN THE
% SOFTWARE.


\documentclass[xetex,serif,xcolor=x11names,compress,handout]{beamer}
\usepackage{pgfpages}
\usetheme{clean}

% Sets title, author, etc.
\title[Title]{Title of My Presentation}
\subtitle{Subtitle}
\author[Klapheke]{Alexander~Klapheke}
\institute[Harvard]{Harvard~University\\\includegraphics[scale=0.4]{shield.pdf}}
\date{November 14, 2012}

% font
\usepackage{fontspec}
\usepackage[libertine]{newtxmath}
\setromanfont[Mapping=tex-text,Numbers=Lining]{Linux Libertine O}
\setmonofont[Scale=MatchLowercase]{Inconsolata Plus 0}

% Strongly discourage hyphenation
\hyphenpenalty=5000
\tolerance=1000

% citations
\usepackage[round,sort&compress]{natbib}
\usepackage{bibentry} % inline references make a reference slide unnecessary
\newcommand\incite[1]{\hfill\textcolor{CustomGray}{\scriptsize #1}} % inline citations

% nice-looking tables
\RequirePackage{booktabs}
% tables with relative fixed-width. use code below to allow wrapping
\RequirePackage{tabularx}
\newcolumntype{C}{>{\centering\arraybackslash}X}
\newcolumntype{L}{>{\raggedright\arraybackslash}X}

% highlighting
\newcommand\hla[1]{\textcolor{CustomRed}{#1}}  % primary color
\newcommand\hlb[1]{\textcolor{CustomBlue}{#1}} % secondary color

%%%%%%%%%%%%%%%%%%%%%%%%%%%%%%%%%%%%%%%
% Linguistics-specific things follow: %
%%%%%%%%%%%%%%%%%%%%%%%%%%%%%%%%%%%%%%%

% glosses
\usepackage{gb4e}
\primebars % use primes instead of 
\let\eachwordone=\it % italicize original in glosses
\resetcounteronoverlays{exx} % don't increment example numbers on \pause

% for trees and graphs
\usepackage{tikz,tikz-qtree}

% Uncomment to start in presentation mode automatically
\hypersetup{%
	colorlinks=false,
	% pdfpagemode=FullScreen,
}

\begin{document}
\bibliographystyle{linquiry2}
\nobibliography{references}

\section{}
\begin{frame}
	\titlepage
\end{frame}

\section{Section}
\subsection{Formatting}
\begin{frame}{Keep information sparse}

	Important information can be \alert{highlighted} to draw the eye to it.

	\vfill

	Make references inline to avoid a cluttered bibliography slide.\\ \incite{Chomsky, N. \textit{Aspects} (MIT, 1965)}

	\vfill

\end{frame}

\subsection{Gloss}
\begin{frame}{Highlight key elements}
	\begin{exe}
		\ex{\label{ex:french} {\glll
				Sont des \hla{mots} qui vont très \hlb{bien} ensemble. \\
				sɔ̃   de  \hla{mo}   ki  vɔ̃   tʁɛ  \hlb{bjɛ̃}  ɑ̃sɑ̃bl     \\
				be.\textsc{3p} \textsc{part} \hla{word-\textsc{pl}} \textsc{rel.nom} go.\textsc{3pl} very \hlb{well} together \\
			\trans `These are words that go together well.'}
		}
	\end{exe}
\end{frame}

\subsection{Formula}
\begin{frame}{Break up long formulas}
	\begin{center}
		\begin{tabular}{cc@{}c@{}cc}
			\Large \colorbox{CustomPale}{\strut$\textsc{be}$} & \Large $=$ &
			\Large \colorbox{CustomPale}{\strut$\text{λ}Q\textcolor{CustomGray}{_{\langle\langle\alpha,t\rangle,t\rangle}}$} & \Large $\,.\,$ &
			\Large \colorbox{CustomPale}{\strut$\text{λ}x\textcolor{CustomGray}{_\alpha}$ $\,.\,$
			\strut$Q\left(\text{λ}y\textcolor{CustomGray}{_\alpha}\,.\,y=x\right)$} \\
			& & Quantified predicate & & Property of being that predicate \\
		\end{tabular}
	\end{center}
\end{frame}

\subsection{Tree}
\begin{frame}{Keep trees simple}
	\begin{center}
		\resizebox{!}{\textheight}{%
		\begin{tikzpicture}
			\Tree
			[.CP
				[.DP \edge[roof]; \node(what){\hla{What}}; ]
				[.C$'$
					[.C \node(are){are}; ]
					[.TP
						[.DP \edge[roof]; \node(you){you}; ]
						[.T$'$
							[.T \node(aret){$t$}; ]
							[.VP
								[.DP \edge[roof]; \node(yout){$t$}; ]
								[.V$'$
									[.V doing? ]
									[.DP \edge[roof]; \node(whatt){\hla{$t$}}; ]
								]
							]
						]
					]
				]
			]
			\hla{\draw[semithick,->,out=225,in=270,looseness=1.25,shorten >=1pt,shorten <=1pt] (whatt.south west) to (what.south);}
			\draw[semithick,->,out=225,in=270,looseness=1.25,shorten >=1pt,shorten <=1pt] (yout.south west)  to (you.south);
			\draw[semithick,->,out=225,in=270,looseness=1.25,shorten >=1pt,shorten <=1pt] (aret.south west)  to (are.south);
		\end{tikzpicture}
	}
	\end{center}
\end{frame}

\subsection{Table}
\begin{frame}{Keep tables simple}
	\newcommand\tablep{\textcolor{CustomRed}{\Large\textbf{+}}}
	\newcommand\tablem{\textcolor{CustomGray}{\Large\textbf{−}}}
	\begin{tabularx}{\linewidth}{lCCCC} \toprule
		Verb type      & Stative & Durative & Telic   & Subinterval \\ \midrule
		State          & \tablep & \tablep  & \tablem & \tablep     \\
		Activity       & \tablem & \tablep  & \tablem & \tablep     \\
		Accomplishment & \tablem & \tablep  & \tablep & \tablem     \\
		Achievement    & \tablem & \tablem  & \tablep & \tablem     \\
		Semelfactive   & \tablem & \tablem  & \tablem & \tablem     \\ \bottomrule
	\end{tabularx}
\end{frame}

\subsection{Graph}
\begin{frame}{Keep graphs simple}
	\begin{center}
		\begin{tikzpicture}[thick,color=CustomDarkBlue]
			% Axes
			\draw[->,color=CustomGray] (0,0) -- coordinate (x axis mid)(9.5,0);
			\draw[->,color=CustomGray] (0,0) -- coordinate (y axis mid)(0,5.5);
			
			% Tick marks
			\foreach \x in {1,...,9}
				\draw[color=CustomGray] (\x cm,1pt) -- (\x cm,-3pt) node[anchor=north] {\x};
			\foreach \y in {0,...,5}
				\draw[color=CustomGray] (1pt,\y cm) -- (-3pt,\y cm) node[anchor=east] {\y};

			% Labels
			\node[below=0.5cm] at (x axis mid) {Time};
			\node[rotate=90,above=0.5cm] at (y axis mid) {Data};

			% \draw[color=CustomRed,ultra thick] plot[smooth,domain=0.363361:10] ({\x},{0.0415256*pow(\x,3)-0.617731*pow(\x,2)+2.67325*\x-0.891787});
			\draw[color=CustomRed,ultra thick] plot[smooth,domain=0:10] ({\x},{0.305827*\x+1.07987});
			\foreach \n in {(1, 1.3087), (2, 2.1548), (3, 2.6518), (4, 2.636), (5, 2.4612), (6, 1.2886), (7, 2.2049), (8, 2.8100), (9, 5.965)}{\node at \n {\LARGE\textbullet};}
			\node[color=CustomRed] at (5,5) {$R^2=0.37$};
		\end{tikzpicture}
	\end{center}
\end{frame}

\section{}
\begin{frame}
	\begin{center}
		{\Huge Thank you!}
	\end{center}
\end{frame}

\end{document}
